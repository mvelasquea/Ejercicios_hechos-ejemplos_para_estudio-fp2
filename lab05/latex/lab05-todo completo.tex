%package list
\documentclass{article}
\usepackage[top=3cm, bottom=3cm, outer=3cm, inner=3cm]{geometry}
\usepackage{multicol}
\usepackage{graphicx}
\usepackage{url}
%\usepackage{cite}
\usepackage{hyperref}
\usepackage{array}
%\usepackage{multicol}
\newcolumntype{x}[1]{>{\centering\arraybackslash\hspace{0pt}}p{#1}}
\usepackage{natbib}
\usepackage{pdfpages}
\usepackage{multirow}
\usepackage[normalem]{ulem}
\useunder{\uline}{\ul}{}
\usepackage{svg}
\usepackage{xcolor}
\usepackage{listings}
\lstdefinestyle{ascii-tree}{
    literate={├}{|}1 {─}{--}1 {└}{+}1 
  }
\lstset{basicstyle=\ttfamily,
  showstringspaces=false,
  commentstyle=\color{red},
  keywordstyle=\color{blue}
}
%\usepackage{booktabs}
\usepackage{caption}
\usepackage{subcaption}
\usepackage{float}
\usepackage{array}

\newcolumntype{M}[1]{>{\centering\arraybackslash}m{#1}}
\newcolumntype{N}{@{}m{0pt}@{}}


%%%%%%%%%%%%%%%%%%%%%%%%%%%%%%%%%%%%%%%%%%%%%%%%%%%%%%%%%%%%%%%%%%%%%%%%%%%%
%%%%%%%%%%%%%%%%%%%%%%%%%%%%%%%%%%%%%%%%%%%%%%%%%%%%%%%%%%%%%%%%%%%%%%%%%%%%
\newcommand{\itemEmail}{mvelasquea@unsa.edu.pe}
\newcommand{\itemStudent}{Mikhail Gabino Velasque Arcos}
\newcommand{\itemCourse}{Laboratorio FUNDAMENTOS DE LA PROGRAMACION II}
\newcommand{\itemCourseCode}{20214260}
\newcommand{\itemSemester}{II}
\newcommand{\itemUniversity}{Universidad Nacional de San Agustín de Arequipa}
\newcommand{\itemFaculty}{Facultad de Ingeniería de Producción y Servicios}
\newcommand{\itemDepartment}{Departamento Académico de Ingeniería de Sistemas e Informática}
\newcommand{\itemSchool}{Escuela Profesional de Ingeniería de Sistemas}
\newcommand{\itemAcademic}{2023 - B}
\newcommand{\itemInput}{Del 29 Setiembre 2023}
\newcommand{\itemOutput}{Al 3 Octubre 2023}
\newcommand{\itemPracticeNumber}{05}
\newcommand{\itemTheme}{Arreglos Bidimensionales de Objetos}
%%%%%%%%%%%%%%%%%%%%%%%%%%%%%%%%%%%%%%%%%%%%%%%%%%%%%%%%%%%%%%%%%%%%%%%%%%%%
%%%%%%%%%%%%%%%%%%%%%%%%%%%%%%%%%%%%%%%%%%%%%%%%%%%%%%%%%%%%%%%%%%%%%%%%%%%%

\usepackage[english,spanish]{babel}
\usepackage[utf8]{inputenc}
\AtBeginDocument{\selectlanguage{spanish}}
\renewcommand{\figurename}{Figura}
\renewcommand{\refname}{Referencias}
\renewcommand{\tablename}{Tabla} %esto no funciona cuando se usa babel
\AtBeginDocument{%
	\renewcommand\tablename{Tabla}
}

\usepackage{fancyhdr}
\pagestyle{fancy}
\fancyhf{}
\setlength{\headheight}{30pt}
\renewcommand{\headrulewidth}{1pt}
\renewcommand{\footrulewidth}{1pt}
\fancyhead[L]{\raisebox{-0.2\height}{\includegraphics[width=3cm]{img/logo_episunsa.png}}}
\fancyhead[C]{\fontsize{7}{7}\selectfont	\itemUniversity \\ \itemFaculty \\ \itemDepartment \\ \itemSchool \\ \textbf{\itemCourse}}
\fancyhead[R]{\raisebox{-0.2\height}{\includegraphics[width=1.2cm]{img/logo_abet}}}
\fancyfoot[L]{Estudiante Mikhail Gabino Velasque Arcos}
\fancyfoot[C]{\itemCourse}
\fancyfoot[R]{Página \thepage}

% para el codigo fuente
\usepackage{listings}
\usepackage{color, colortbl}
\definecolor{dkgreen}{rgb}{0,0.6,0}
\definecolor{gray}{rgb}{0.5,0.5,0.5}
\definecolor{mauve}{rgb}{0.58,0,0.82}
\definecolor{codebackground}{rgb}{0.95, 0.95, 0.92}
\definecolor{tablebackground}{rgb}{0.8, 0, 0}

\lstset{frame=tb,
	language=bash,
	aboveskip=3mm,
	belowskip=3mm,
	showstringspaces=false,
	columns=flexible,
	basicstyle={\small\ttfamily},
	numbers=none,
	numberstyle=\tiny\color{gray},
	keywordstyle=\color{blue},
	commentstyle=\color{dkgreen},
	stringstyle=\color{mauve},
	breaklines=true,
	breakatwhitespace=true,
	tabsize=3,
	backgroundcolor= \color{codebackground},
}

\begin{document}
	
	\vspace*{10px}
	
	\begin{center}	
		\fontsize{17}{17} \textbf{ Informe de Laboratorio 05 }
	\end{center}
	\centerline{\textbf{\Large Tema: Arreglos Bidimensionales de Objetos}}
	%\vspace*{0.5cm}	

	\begin{flushright}
		\begin{tabular}{|M{2.5cm}|N|}
			\hline 
			\rowcolor{tablebackground}
			\color{white} \textbf{Nota}  \\
			\hline 
			     \\[30pt]
			\hline 			
		\end{tabular}
	\end{flushright}	

	\begin{table}[H]
		\begin{tabular}{|x{4.7cm}|x{4.8cm}|x{4.8cm}|}
			\hline 
			\rowcolor{tablebackground}
			\color{white} \textbf{Estudiante} & \color{white}\textbf{Escuela}  & \color{white}\textbf{Asignatura}   \\
			\hline 
			{\itemStudent \par \itemEmail} & \itemSchool & {\itemCourse \par Semestre: \itemSemester \par Código: \itemCourseCode}     \\
			\hline 			
		\end{tabular}
	\end{table}		
	
	\begin{table}[H]
		\begin{tabular}{|x{4.7cm}|x{4.8cm}|x{4.8cm}|}
			\hline 
			\rowcolor{tablebackground}
			\color{white}\textbf{Laboratorio} & \color{white}\textbf{Tema}  & \color{white}\textbf{Duración}   \\
			\hline 
			\itemPracticeNumber & \itemTheme & 04 horas   \\
			\hline 
		\end{tabular}
	\end{table}
	
	\begin{table}[H]
		\begin{tabular}{|x{4.7cm}|x{4.8cm}|x{4.8cm}|}
			\hline 
			\rowcolor{tablebackground}
			\color{white}\textbf{Semestre académico} & \color{white}\textbf{Fecha de inicio}  & \color{white}\textbf{Fecha de entrega}   \\
			\hline 
			\itemAcademic & \itemInput &  \itemOutput  \\
			\hline 
		\end{tabular}
	\end{table}
	
	\section{Actividades}
	\begin{itemize}		
		\item Cree un Proyecto llamado Laboratorio5
\item Usted deberá crear las dos clases Soldado.java y VideoJuego2.java. Puede reutilizar lo
desarrollado en Laboratorio 3 y 4.
		\item Del Soldado nos importa el nombre, puntos de vida, fila y columna (posición en el tablero).
		\item El juego se desarrollará en el mismo tablero de los laboratorios anteriores. Pero ahora el
tablero debe ser un arreglo bidimensional de objetos.
\itm Inicializar el tablero con n soldados aleatorios entre 1 y 10. Cada soldado tendrá un nombre
autogenerado: Soldado0, Soldado1, etc., un valor de puntos de vida autogenerado
aleatoriamente [1..5], la fila y columna también autogenerados aleatoriamente (no puede
haber 2 soldados en el mismo cuadrado). Se debe mostrar el tablero con todos los soldados
creados (usar caracteres como | _ y otros).

	
	\end{itemize}
		
	\section{SOLUCIONARIO}
	\begin{itemize}
		\item Se hace el uso de arreglo de objetos bidimencionales  como el de soldado.java y VideoJuego_v2.java para completar la actividad como se muestra en la siguiente seccion
	\end{itemize}

	\subsection{CODIGO FUENTE}
	\begin{itemize}	
		\item Se crea la clase soldado.java
		\item Se crea la clase principal:   VideJuego_v2.java
	\end{itemize}	
		
	\begin{lstlisting}[language=bash,caption={Creando la clase Nave}][H]
		vim soldado.java
		  vim VideoJuego_v2.java
	\end{lstlisting}
	
	\begin{lstlisting}[language=bash,caption={Creando la clase soldado}][H]
			
	public class Soldado {
		/*
		 Reusando el codiogo de los anterioes labs
		
			 laboratorio Nro 5 ejercicio 1
			 //clase soldado
			 Autor :Mikhail Gabino Velasque Arcos
			colaboro:---
			tiempo:
			 */
	   private String nombre;
	    private int puntosDeVida;
	    private int fila;
	    private int columna;

	    public Soldado(String nombre, int puntosDeVida) {
	        this.nombre = nombre;
	        this.puntosDeVida = puntosDeVida;
	    }

	    public String getNombre() {
	        return nombre;
	    }

	    public int getPuntosDeVida() {
	        return puntosDeVida;
	    }

	    public void setPuntosDeVida(int puntosDeVida) {
	        this.puntosDeVida = puntosDeVida;
	    }

	    public int getFila() {
	        return fila;
	    }

	    public void setFila(int fila) {
	        this.fila = fila;
	    }

	    public int getColumna() {
	        return columna;
	    }

	    public void setColumna(int columna) {
	        this.columna = columna;
	    }

	    @Override
	    public String toString() {
	        return "Nombre: " + nombre + ", Vida: " + puntosDeVida + ", Fila: " + fila + ", Columna: " + columna;
	    }
	}		
			
			
			
			
			
	\end{lstlisting}	
	\begin{lstlisting}[language=bash,caption={Creando la clase principal de VideoJuego_v2.java}][H]
	import java.util.ArrayList;
import java.util.Collections;
import java.util.Random;
public class VideoJuegoo_v2 {
	 public static void main(String[] args) {
	        int filas = 10; 
	        int columnas = 10; 
	        int cantidad = new Random().nextInt(10) + 1;

	        ArrayList<Soldado> soldados = crearSoldados(cantidad);
	        Soldado[][] tablero = crearTablero(filas, columnas, soldados);  
	        mostrarTablero(tablero);     
	        System.out.println("\nLista de Soldados:");
	        for (Soldado soldado : soldados) {
	            System.out.println("Nombre: " + soldado.getNombre() + ", Vida: " + soldado.getPuntosDeVida() + ", Fila: " + soldado.getFila() + ", Columna: " + soldado.getColumna());
	        }}
	    public static ArrayList<Soldado> crearSoldados(int n) {
	        ArrayList<Soldado> soldados = new ArrayList<>();
	        Random r = new Random();

	        for (int i = 1; i <= n; i++) {
	            String nombre = "Soldado" + i;
	            int puntosDeVida = r.nextInt(10) + 1; 
	            soldados.add(new Soldado(nombre, puntosDeVida));
	        }

	        return soldados;
	    }

	    public static Soldado[][] crearTablero(int filas, int columnas, ArrayList<Soldado> soldados) {
	        Soldado[][] tablero = new Soldado[filas + 1][columnas + 1];
	        Random r = new Random();

	        for (Soldado soldado : soldados) {
	            int fila, columna;
	            do {
	                fila = r.nextInt(filas) + 1; 
	                columna = r.nextInt(columnas) + 1; 
	            } while (tablero[fila][columna] != null);

	            soldado.setFila(fila);
	            soldado.setColumna(columna);
	            tablero[fila][columna] = soldado;
	        }

	        return tablero;
	    }

	    public static void mostrarTablero(Soldado[][] tablero) {
	        for (int fila = 1; fila < tablero.length; fila++) { 
	            for (int columna = 1; columna < tablero[fila].length; columna++) { 
	                Soldado soldado = tablero[fila][columna];
	                if (soldado != null) {
	                    System.out.print("| " + soldado.getNombre().substring(7) + " |");
	                } else {
	                    System.out.print("|___|");
	                }
	            }
	            System.out.println();
	        }
	    }
	}
	
	
				\end{lstlisting}
			
\end{document}